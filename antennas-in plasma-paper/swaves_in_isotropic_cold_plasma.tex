%% Advances in Space Research
% August 2010
% 
% Template article for preprint document class 'elsarticle'
% with harvard style bibliographic references
%
% NB: elsarticle includes natbib package; for more information, cf. http://www.elsevier.com/wps/find/authorsview.authors/elsarticle
% 
% Copyright � 2010 Elsevier B.V. All rights reserved.

%% Document class
\documentclass[preprint,authoryear,12pt]{elsarticle}

% Use the following command for final-print formatting
% \documentclass[final,authoryear,5p]{elsarticle}

%% Figures packages
% If you use PostScript figures in your article
% use the graphics package for simple commands
% \usepackage{graphics}
% or use the graphicx package for more complicated commands
% \usepackage{graphicx}
% or use the epsfig package if you prefer to use the old commands.
\usepackage{epsfig}

%% Mathematical symbols
% The amssymb package provides various useful mathematical symbols
\usepackage{amssymb}

%% Hyperlinks
\usepackage[ps2pdf,%
a4paper=true,%
breaklinks=true,%
colorlinks=true,%
pdfauthor={T.H. Oswald et al.},%
pdftitle={Inclusion of isotropic plasma effects in the numerical calibration of the STEREO/WAVES antennas}%
]{hyperref}

%% Journal ID
\journal{Advances in Space Research}

\begin{document}

%%%%%%%%%%%%%%%%%%%%%%%%%%%%%%%%%%%%%%%%%%%%%%%%%%%%%%%%%%%%%%%%%%%%%%%%%%%%%
%% Frontmatter
\begin{frontmatter}

%% Title, authors and addresses

% Use the tnoteref command within \title and fnref within \author or \address for footnotes;
% use the corref command within \author for corresponding author footnotes;
% use the ead command for the email address,
% and the form \ead[url] for the home page:
% \title{Title\tnoteref{label1}}
% \tnotetext[label1]{}
% \author{Name\corref{cor1}\fnref{label2}}
% \ead{email address}
% \ead[url]{home page}
% \fntext[label2]{}
% \cortext[cor1]{}
% \address{Address\fnref{label3}}
% \fntext[label3]{}



\title{Inclusion of isotropic plasma effects in the numerical calibration of the STEREO/WAVES antennas}


% Use optional labels to link authors explicitly to addresses:
% \author[label1,label2]{}
% \address[label1]{}
% \address[label2]{}

\author{T.H. Oswald\corref{cor}}
\address{Space Research Institute, Austrian Academy of Sciences, Schmiedlstrasse 6, Graz, A-8042, Austria.}
\cortext[cor]{Corresponding author}

\ead{thomas.oswald@aeroware.at}

% Url can be given like this:
% \ead[url]{http://www.elsevier.com/wps/find/authorsview.authors/latex}


\author{H. O. Rucker and W. Macher and G. Fischer and M. Sampl}
\address{Space Research Institute, Austrian Academy of Sciences, Schmiedlstrasse 6, Graz, A-8042, Austria.}



\begin{abstract}
The STEREO/WAVES experiment uses three orthogonal antennas to receive radiation and plasma waves created by natural processes. For a correct interpretation of the received radiation the antenna properties have to be known very accurately, so these antennas have to be calibrated. There are 2 major items which influence antenna properties in a way that they can not be predicted without analysis. One major influence is the irregular shape of the spacecraft body, which is coated with a conductive material and therefore acting as part of the antenna itself. The other major influence comes from the space plasma. A numerical calibration of the STEREO/WAVES antennas was published in \citep{ossi09}, while the results of an experimental calibration can be found in \citep{macher07}. During these calibrations, the surrounding space plasma was ignored and vacuum was assumed. In this publication, the effect of isotropic cold plasma is incorporated into the numerical calculations and the results are compared with existing values.
\end{abstract}

\begin{keyword}
%first keyword \sep second keyword \sep more keywords
radio science; plasma; antenna; radio experiment; spacecraft; Stereo; waves; solar radiation;
% keywords here, in the form: keyword \sep keyword
% PACS codes here, in the form: \PACS code \sep code
\end{keyword}

\end{frontmatter}

\parindent=0.5 cm

%%%%%%%%%%%%%%%%%%%%%%%%%%%%%%%%%%%%%%%%%%%%%%%%%%%%%%%%%%%%%%%%%%%%%%%%%%%%%
%% Main text
\section{Introduction}


You can simply replace the text/figures/tables in this file
with your own material and process it to produce a referee version
or camera ready version of your manuscript.

\section{An equation}

  \begin{equation}
    \label{eq:1}
 {%
    \sum_{i=0}^{\infty}A^n\int \mathrm{d}x\, \frac{F_n(x)}{A_n + B_n} =
    B^n C^n \int\mathrm{d}x\,\int \mathrm{d}y\,
    \frac{G_n(x,y)}{\mathcal{A}_n{x} + \mathcal{B}_n{y}}
    }
%  \overfullrule 5pt
%  \mathindent\linewidth\relax
%  \advance\mathindent-259pt
  \end{equation}

\section{Figures}

\begin{figure}
\label{figure1}
\begin{center}
%\includegraphics*[width=10cm,angle=-90]{figure1.ps}
\end{center}
\caption{This figure shows ......}
\end{figure}

As shown in this example, you can provide figures embedded in the text.
Alternately you can provide all the figures at the end with a separate page
giving the figure numbers and corresponding captions. Final placement
of the figures and tables will be done by the typesetters.



%%%%%%%%%%%%%%%%%%%%%%%%%%%%%%%%%%%%%%%%%%%%%%%%%%%%%%%%%%%%%%%%%%%%%%%%%%%%%
%% Appendices
% The Appendices part is started with the command \appendix;
% appendix sections are then done as normal sections
% \appendix

\begin{thebibliography}{}

% \bibitem[Names(Year)]{label} or \bibitem[Names(Year)Long names]{label}.
% (\harvarditem{Name}{Year}{label} is also supported.)
% Text of bibliographic item

\bibitem[{\textit{Davidson}(2005)}]{davidson}
Davidson, B. (2005), \textit{Computational Electromagnetics for RF and
  Mictrowave Engineering}, Cambridge University Press.

\bibitem[{\textit{Ginzburg}(1970)}]{ginzburg}
Ginzburg, V. (1970), \textit{The propagation of electromagnetic waves in
  plasma, Second Edition}, Pergamon Press.

\bibitem[{\textit{Harrington}(1968)}]{harrington}
Harrington, R. (1968), \textit{Field Computation by Moment Methods}, Robert E.
  Krieger Publishing Company.

\bibitem[{\textit{Macher et~al.}(2007)\textit{Macher, Oswald, Fischer, and
  Rucker}}]{macher07}
Macher, W., T.~Oswald, G.~Fischer, and H.~Rucker (2007), Rheometry of
  multi-port spaceborn antennas including mutual antenna capacitances and
  application to stereo/waves., \textit{Meas. Sci. Technol.}, \textit{18},
  3731--3742.

\bibitem[{\textit{Melrose}(1980)}]{melrose1}
Melrose, D. (1980), \textit{Plasma Astrophysics: Nonthermal Processes in
  Diffuse Magnetized Plasma}, Gordon and Beach Science Publishers.

\bibitem[{\textit{Oswald et~al.}(2009)\textit{Oswald, Macher, Rucker, Fischer,
  Taubenschuss, Bougeret, Lecacheux, Kaiser, and Goetz}}]{ossi09}
Oswald, T., W.~Macher, H.~Rucker, G.~Fischer, U.~Taubenschuss, J.~Bougeret,
  A.~Lecacheux, M.~Kaiser, and K.~Goetz (2009), Various methods of calibration
  of the stereo/waves antennas, \textit{Adv. Space Res.},\textit{43},
  355--364.

\bibitem[{\textit{Panchenko et~al.}(2010)\textit{Panchenko, Rucker, Macher,
  Cecconi, Oswald, and Fischer}}]{panchenko10}
Panchenko, M., H.~Rucker, W.~Macher, B.~Cecconi, T.~Oswald, and G.~Fischer
  (2010), Stereo/waves antennas calibrated by akr, in \textit{European
  Planetary Science Congress 2010}, vol.~5.

\bibitem[{\textit{Richmond}(1966)}]{richmond66}
Richmond, J. (1966), A wire-grid model for scattering by conducting bodies,
  \textit{IEEE Transactions on Antennes and Propagation}, \textit{14},
  782--786.

\bibitem[{\textit{Stix}(1992)}]{stix}
Stix, T. (1992), \textit{Waves in Plasmas}, American Institute of Physics.

\end{thebibliography}

%\clearpage

\begin{table}
\caption{This is the caption of this table}
\begin{tabular}{ll}
\hline
Parameter&Value\\
\hline
Parameter 1 & $526.849 \pm 0.003$ s \\
Parameter 2 & $5268.49 \pm 0.03$ s \\
Parameter 3 & $52684.9 \pm 0.3$ s \\
\hline
\end{tabular}
\label{table1}
\end{table}

\end{document}
